\documentclass[a4paper, 12pt]{article}
\input{hw-settings.tex}

\begin{document}
\begin{center}
{\fs{15}\bfseries {计算机动画原理与技术 ~作业 3 ~报告}}

\vspace{0.5\baselineskip}

{\fs{14} \kaishu 于泽汉 \hspace{1em} \textsf{No.118039910141}}
\end{center}

\noindent 在本次实验中,对于三种不同的数值积分方法,可以分析观察得到如下结论:

\begin{itemize}[leftmargin=2em, label={}]
\item \textbf{常数加速:}显式欧拉法的常数加速效果较差;隐式欧拉法的常数加速效果也较差,因为它和显式的一样都是一阶的;梯形法则的常数加速效果很好,因为它是二阶的。

\item \textbf{位置依赖性:}显式欧拉法的位置依赖性很差,受不同参数变化影响较大,因此很不稳定;隐式欧拉法的位置依赖性稍好一些,可以保证稳定,但是受到一定程度的衰弱;梯形法则的位置依赖性最佳,不仅稳定,而且没有衰弱。

\item \textbf{速度依赖性:}显式欧拉法的速度依赖性较好,在满足一定条件下是稳定的;隐式欧拉法的速度依赖性很好,不过是单调的;梯形法则的速度依赖性最佳,稳定非单调。

\item \textbf{运算速度:}显式欧拉法的运算速度最快;隐式欧拉法和梯形法则的运算速度要慢一些。

\item \textbf{易实现性:}显式欧拉法最容易实现;隐式欧拉法和梯形法则都要复杂很多,并且实现时很容易出错。

\end{itemize}

\vspace{\baselineskip}

\begin{itemize}[leftmargin=*, label={}]

\item 三种方法的实现效果和对应代码见 \texttt{hw3.html}。\\
浏览器(推荐使用 Chrome)打开可查看动画,文本编辑器打开可查看代码。\\

本次实验实现的功能包括:

\begin{enumerate}[leftmargin=*]

\item 三种不同的数值积分方法
\item 自由调整各种参数
\item 拖拽任意节点并实时反馈

\end{enumerate}

\begin{center}
\includegraphics[width=0.8\textwidth]{images/euler.png}\\
网格弹簧质点系统的模拟仿真以及用户界面
\end{center}

\end{itemize}

\end{document}