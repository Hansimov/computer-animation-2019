\documentclass[a4paper, 12pt]{article}
\input{hw-settings.tex}

\begin{document}
\begin{center}
{\fs{15}\bfseries {计算机动画原理与技术 ~作业 2 ~报告}}

\vspace{0.5\baselineskip}

{\fs{14} \kaishu 于泽汉 \hspace{1em} \textsf{No.118039910141}}
\end{center}

在本次实验中,对于三种不同的数值积分方法,可以分析观察得到如下结论:

\begin{itemize}[leftmargin=2em, label={}]
\item \textbf{精度:}欧拉法的精度较差,而且当步长提高时,误差也会增加;中点法精度稍高一些,误差相较欧拉法要小很多;龙格-库塔法的精度最高。

\item \textbf{稳定性:}欧拉法的稳定性较差,受不同参数变化影响较大;中点法稳定性稍好一些,积分结果的波动随参数变化相对较小;龙格-库塔法的稳定性最好,不同参数对积分结果的影响很小。

\item \textbf{运算速度:}欧拉法最快,中点法其次,龙格-库塔法最慢。

\end{itemize}

\vspace{\baselineskip}

\textbf{\fs{15}A. Ballistic Motion}

\begin{itemize}[leftmargin=2em, label={}]

\item 三种方法的实现效果和对应代码见 \texttt{hw2\_ballistic\_motion.html}。\\
浏览器(推荐使用 Chrome)打开可查看动画,文本编辑器打开可查看代码。\\
这里为了突出效果,步长选取为 0.02。

\begin{center}
\includegraphics[width=\textwidth]{images/ballistic_motion.png}\\
三种不同的数值积分方法对应的炮弹抛射轨迹
\end{center}

\end{itemize}


\textbf{\fs{15}B. Spring-Mass Simulator}

\begin{itemize}[leftmargin=2em, label={}]

\item 三种方法的实现效果和对应代码见 \texttt{hw2\_spring\_mass.html}。\\
浏览器(推荐使用 Chrome)打开可查看动画,文本编辑器打开可查看代码。\\
这里为了突出效果,步长选取为 0.05。\\
这里选取了两组不同的 $k$ 和 $m$ 以观察效果。

\begin{center}
\includegraphics[width=\textwidth]{images/spring_mass.png}\\
三种不同的数值积分方法对应的弹簧-质量系统
\end{center}


\end{itemize}

\end{document}